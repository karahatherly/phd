\chapter{Conclusion}
\label{chap:conclusion}

Emerging cyber-physical systems represent an opportunity for greater safety, security and 
automation, as they can replace systems where human errors prevail. While humans are excellent at
innovation and creativity, they cannot compete with machines that do not get tired, drunk or
distracted when completing
monotonous, repetitive tasks. Car accidents are overwhelmingly 
caused by human error, measured at 94\% in the US~\citep{Singh_15}, something that self-driving cars can overcome. Autonomous aircraft and other fleet vehicles, smart cities and smart factories hold similar promise. 

As established in \cref{sec:motivation}, in order to practically develop such systems,
they must be mixed-criticality, as certification of all parts of such a system to high-assurance
levels is not feasible. Consequently, mixed-criticality systems require strong temporal isolation
and asymmetric protection between sub-systems of different criticality levels, 
which must also be able to safely share resources.

This thesis has drawn on real-time theory and systems practice to develop core mechanisms required
in a high-assurance, trusted computing base for mixed criticality systems. Importantly, the
mechanisms we have developed do not violate other requirements, like policy-freedom, integrity,
confidentiality, spatial isolation, and security. 

In \cref{chap:model}, we introduced our model for treating time as a first-class resource, and 
defined a minimal set of
mechanisms that can be leveraged to implement temporal isolation between threads, even within shared
servers. Our primitives also allow for more complex, application specific user-level schedulers to
be constructed efficiently, as we demonstrated in the evaluation. Significantly, we provide a
new model for L4 kernels to manage time as a first-class resource, without restricting policy
options including overbooking, 
or requiring specific correlations between the priority of a task and the scheduling model, or
importance of that task. 
We integrate a capability system with the non-fungible resource that is time, without violating our
existing isolation and confidentiality properties, or requiring a heavy performance overhead.
Our mechanisms provide a basic scheduler, which is a critical part of the
trusted computing base, and allow for more system-specific schedulers to be constructed at
user-level. Additionally, we do not force specific
resource-sharing prioritisation policies, but provide mechanisms to allow for their implementation at user-level. 
\clearpage

\section{Contributions}

Specifically, we make the following contributions:

\begin{itemize}
    \item Mechanisms for principled management of time through capabilities to scheduling contexts,
        which are compatible with the fast IPC implementations traditionally used in high-performance
        microkernels, and also compatible with established real-time resource-sharing policies;
    \item an implementation of those mechanisms in the non-preemptible \selfour microkernel, and an
        exploration of how the implementation interacts with the existing kernel model;
    \item an evaluation of the overheads and isolation properties of the implementation, including
        a demonstration of isolation in a shared server through timeout-fault handling;
    \item a design and implementation of many different, user-level timeout-fault handling policies;
    \item and an implementation of a user-level \gls{EDF} scheduler which is as fast as the
        \litmus, in-kernel EDF scheduler, which shows that despite the fixed-priority scheduler
        present in the kernel, other schedulers remain practical;
    \item and a design and implementation of a criticality switch at user-level, which shows that
        criticality is not required to be a kernel-provided property.
\end{itemize}

The implementation is complete, and the verification of this implementation is currently in progress.
Specifications have already been developed by the verification engineering team at Data61, CSIRO,
who are now completing the first-level refinement of the functional correctness proof.
The maximum controlled priority feature has already been verified and merged to the master kernel. 
Verification is beyond the scope of this PhD, although we continue to work closely with
the team to assist in verification. 

This work is part of
the Side-Channel Causal Analysis for Design of Cyber-Physical Security project for the U.S
Department of Homeland Security.

Additionally, during the development of this thesis we have made extensive contributions to the 
\selfour benchmarking suite, testing suite and user-level libraries, all of which provide better 
support for running experiments and building systems on \selfour. 

\section{Future work}

Modern \glspl{CPU} have dynamic frequency scaling and low-power modes in order to reduce energy usage, which
is of high concern in many embedded systems. The implementation as it stands assumes a constant
\gls{CPU} frequency, and all kernel operations are calculated in \gls{CPU} cycles. Frequency scaling
can have undesirable effects on real-time processes: if the period is specified in microseconds, then
converted to cycles and the CPU frequency changes, does the period remain correct? This is a
limitation of our model and promising area for future work.

Although we provide a mechanism for cross-core IPC, where passive threads migrate between processing
cores and active threads remain fixed, there is much more to consider in terms of multicore
scheduling and resource sharing. Our model provides a partitioned, fixed priority scheduler, and 
load balancing can be managed by user level. Further experiments to evaluate the mechanisms for higher-level
multicore scheduling and resource sharing protocols is required. 

Finally, this work does not consider counter-measures for timing-related covert and side channels,
security risks which arise from
shared hardware~\citep{Ge_YCH_18}. \emph{Covert channels} are unintentional, explicit communication
channels which may be used by a malicious sender in a trusted component to leak sensitive data to a receiver
in an untrusted component. Providing a time-source can open a covert channel, where the sender
manipulates timing behaviour through shared architectural structures like caches, and the receiver
uses the time source to observe the behaviour. Shared hardware also presents the risk of
\emph{side channels}, 
where an attacker in an untrusted
component can learn secret information about a trusted component by observing timing behaviour. 
Both attack vectors are risks in mixed-criticality systems, which we do not consider in this
thesis, and merit extensive future work. 

\section{Concluding Remarks}

Original automobiles did not have seat belts, as safety was never a concern. Perhaps one day we
will reflect on human-piloted, high-speed vehicles in the same way. For this future to eventuate, 
cyber-physical systems which combine components of different criticality on shared hardware are essential.
Our work has focused on principled mechanisms for managing time in such systems,
making one significant step towards a future of trustworthy,
safe and secure autonomous systems.

\bigskip
This material is based on research sponsored by the Department of Homeland Security (DHS) Science
and Technology Directorate, Cyber Security Division (DHS S\&T/CSD) BAA HSHQDC-14-R-B00016, and the
Government of United Kingdom of Great Britain and the Government of Canada via contract number
D15PC00223. 
