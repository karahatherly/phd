\chapter{Conclusion}
\label{chap:conclusion}

\section{Contributions}

Specifically, we make the following contributions:
\begin{itemize}
\item A capability system for time that has low overhead and does not
  limit the system to a particular scheduling policy, including
  implementation of arbitrary scheduling policies at user level;
\item a notion of CPU reservations that is compatible with
  fast IPC implementations traditionally used in high-performance
  microkernels, and is compatible with established real-time resource-sharing policies;
\item the first OS kernel supporting an explicit notion of
  criticality, orthogonal to priority, in addition to the above;
\item an exploration of implementation in the non-preemptible \selfour
  microkernel and its interaction with the kernel's model of
  user-level management of kernel memory, which is a critical enabler
  of strong spatial isolation.
\end{itemize}

\section{Future work}

\TODO{Frequency scaling}
\TODO{multicore locking and scheduling}


This chapter outlines what this PhD project has accomplished so far and what I plan to achieve in the future.

\section{Progress}

This research will investigate the scheduling and resource sharing problems of running application software at different criticality levels with different scheduling requirements in the context of the verified microkernel, \selfour.
This research will attempt to break new ground by developing a real-time operating system kernel with full system schedulability analysis, including that of the kernel code and scheduler.

The goals of this project are to provide (reproduced from the introduction):

\begin{enumerate}[label=\textbf{G\arabic*}]
  \item \label{G1} A principled approach to processor management, treating time as a fundamental kernel resource, while allowing it to be overbooked, a key requirement of mixed-criticality systems;
  \item \label{G2} safe resource sharing between applications of different criticalities and different temporal requirements.
\end{enumerate}

This research is motivated by the rise in demand for mixed-criticality systems.
It is clear from \Cref{chap:background} that although much work has been conducted on scheduling algorithms for mixed-criticality systems, little work has considered the implications for operating systems.
In addition, mixed-criticality systems require more strict access control to time than has typically been required from standard real-time systems.
This motivates \ref{G1}, which investigates how trusted applications can control the use of time as a resource, such that all timing requirements are met and low criticality tasks cannot compromise the timing requirements of higher criticality tasks.
\ref{G2} covers how time flows through shared resources, which may be shared between tasks of different criticalities and different temporal strictness (\gls{HRT}, \gls{SRT}, best effort) without violating the real-time requirements.

This research is split into several clear phases: background research (year 1), design and implementation (year 2) and evaluation (year 3).
We are currently at the middle of the evaluation phase, having completed the following:
:
\begin{itemize}
	\item developed primitives in \selfour to allow a diverse range of scheduling options at user-level;
	\item extended the \selfour capability model to apply to time, such that it can be enforced by the kernel;
	\item evolved the current \selfour \gls{IPC} mechanisms to allow for analysable blocking of real-time tasks, as required for mixed-criticality resource sharing.
        \item extended the model to cover and run on \gls{SMP} systems.
        \item built a set of microbenchmarks for evaluating \selfour master against the real-time version.
        \item built several examples including an AES encryption server, and built a variety of timeout exception handling 
\end{itemize}

Throughout the evaluation, I've discovered problems with the design that have faciliated entire redesigns, tweaks and revaluations. 

The next stage is run the benchmark suite on the newest kernel and complete the case studies.

\subsection{Ongoing Evaluation}

Our design and will be evaluated according to the following qualities over the next 12 months:

\subsubsection{Policy-Mechanism Separation}

The aim of the project is to be as true to the microkernel minimality principle as possible -- this means that kernel features should avoid influencing user policy decisions as much as possible.
The motivation for using the microkernel minimality policy is to provide widely-applicable, trustworthy scheduling.
Our case studies should allow us to evaluate how successful our design is in keeping policy and mechanism separate.

\subsubsection{Schedulability analysis}

Designs must be conscious of minimising actual and estimated \gls{WCET} as in order to provide a kernel that allows for schedulability analysis, a full profile of \gls{WCET} must be available.
At the end of the project it should be possible to take a full description of a closed system and irrefutably show that the system is always schedulable.
Similarly, for a mixed-criticality system, it must be possible to show that the critical parts of the system are always schedulable.

\subsubsection{Measures of verification potential}

I have determined that the code follows the verification subset of C and passes through the verification C-Parser.
I am actively working with the verification team this year to verify the first set of patches. 

\subsection{Case studies}

Case studies will demonstrate that the scheduling and resource sharing mechanisms are practical.
So far I have built user level schedulers, an AES server for measuring rollback costs. 
I need to finish a network case study and the SMACCM case study. 

\subsubsection{SMACCM}

Another ideal case study for mixed-criticality systems is an autonomous helicopter.
Our research group is currently working on building a high-assurance system for such a helicopter (\url{www.ssrg.nicta.com.au/projects/TS/SMACCM}), which I should be able to test running my scheduler.
I have already been involved with the SMACCM project, advising on real-time scheduling, resource sharing and analysis.

\section{Timeline}

I have completed 3 years of my PhD project, with 1 year remaining.
I expect to safely complete in by the end of 2017.
Below is a timeline to completion.

\begin{center}
\begin{tabular}{|c|p{0.7\textwidth}|}\hline
March & Get microbenchmark suite running on latest kernel API, finish write up of all microbenchmarks to date. \\\hline
April & Design network case study demonstrating temporal isolation. Write up. \\\hline
May & Finish implementatio and model write up. Start quadcopter/SMACCM case study. \\\hline
June & Finish quadcopter/SMACCM case study, write up. \\\hline
July & Work on multicore resource sharing policies. \\\hline
August & Work on papers and phd. \\\hline
September & Work on papers and phd. \\\hline
October & Submit paper on microkernel design to Eurosys, RT multicore resource sharing paper to RTAS. \\\hline
November & Finalise writing. \\\hline
December & Send out phd for feedback, submit once collected and integrated. \\\hline
\end{tabular}
\end{center}

\subsection{Community Engagement}

I participated in the \gls{SOSP} 2013 Diversity Workshop presenting a pitch of my PhD topic centered around a poster.
I presented at the Trustworthy Systems Summer School in 2017 and 2015.

\subsection{Publications}

In a collaboration with Manohar Vanga, Felipe Cerqueira and Bj\"{o}rn Brandenburg from the \gls{MPISS}, my supervisor Gernot Heiser and I submitted a paper to the RTSS, during the first year of my PhD.
This paper was rejected due to lack of theoretical novelty, which we put down to not being a good fit for that particular conference.

This paper was rejected again, from \gls{RTAS}, so we ultimately decided that it was not a good fit for the real time community.
We plan a resubmission of this paper to a systems conference once one of the case studies is complete, as to suit the systems community a demonstration of the work in real world scenario is required.

I have submitted papers on the scheduling  API to OSDI and \gls{RTAS}, however these were rejected.
I plan two submit two papers in the final year, one to Eurosys on capabilities to time, and one to \gls{RTAS} demonstrating the API to build mixed-criticality IPC protocols on multicore.

\subsection{Summary}

Over the past year I have developed a design for a principled approach to time in seL4, involving integrating resource sharing into \gls{IPC}.
I have nearly finished a concrete implementation and the next 12 months will involve evaluation and case studies.
Although there is a long road ahead, I believe I am on track to a timely completion.

\section{Conclusion}

Mixed-criticality real-time systems that provide isolation and hard guarantees at the operating system level are desirable, due to the unscalable nature of hardware isolation.
Much research has been conducted into how to schedule workloads for real-time, mixed-criticality systems, however the implications for operating systems are yet to be examined, with most implementations falling short of what is required in a highly critical environment.

As a result, this ongoing research investigates the implications of mixed-criticality, real-time scheduling in the context of a high-assurance operating system, driven by the rise of mixed-criticality systems in safety-critical environments.
The research investigates the trade-offs between verification potential, performance and policy-mechanism separation with respect to mixed-criticality scheduling and resource sharing.
In addition to a real-time iteration of seL4 that offers full-system schedulability analysis including the kernel and scheduler, the contributions are expected to provide a model for future L4 real-time scheduling primitives and a novel approach to applying fine-grained permissions to time without sacrificing performance.


