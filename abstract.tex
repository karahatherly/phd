\begin{abstract}
Criticality of a software system refers to the severity of the impact of a failure.
In a high-criticality system, failure risks significant loss of life or damage to the environment.
In a low-criticality system, failure may risk a downgrade in user-experience.
As criticality of a software system increases, so too does the cost and time to develop that
software: raising the criticality also raises the assurance level, with the highest levels requiring
extensive, independent certification.

For modern cyber-physical systems, including autonomous aircraft and vehicles, the traditional
approach of isolating systems of different criticality by using completely separate physical
hardware, is no
longer practical, being both restrictive and inefficient.
The result is mixed-criticality systems, where software applications with different criticalities
execute on the same hardware. Sufficient mechanisms are required to ascertain that software in
mixed-criticality systems  is sufficiently isolated, otherwise, all software on that hardware is promoted
to the highest criticality level, driving up costs to impractical levels. For mixed-criticality systems to be
viable, both spatial and temporal isolation are required.

Current aviation standards allow for mixed-criticality systems where temporal and spatial resources
are strictly and statically partitioned, allowing some improvement over 
in time and space. However, further improvements are not possible, but required for future
innovation in cyber-physical systems. 

This thesis explores further operating mechanisms to allow for mixed-criticality
software to share resources in far less restrictive ways, opening further possibilities in
cyber-physical system design without sacrificing assurance properties. 
Two key properties are required: first, time must be managed as a central resource of the system, while allowing for overbooking with asymmetric protection without increasing certification burdens.
Second, components of different criticalities should be able to safely share resources without suffering undue utilisation penalties.
The implementation platform is the \selfour microkernel, which is built for the safety-critical, high assurance domain.

\end{abstract}
\addcontentsline{toc}{chapter}{Abstract}
\clearpage
