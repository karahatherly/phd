

\chapter{Operating Systems}
\label{chap:operating-systems}

In this chapter we provide a survey of existing operating systems, with a focus on how each one
provides mechanisms for mixed criticality systems, including temporal isolation, asymmetric protection,
and resource sharing. 

\section{POSIX}

The \gls{POSIX} standard specifies real-time operating systems interfaces~\citep{Harbour_93} which
influence a large amount of \glspl{OS}.  Scheduling policies specified by \gls{POSIX} are shown in
\Cref{tab:posix-sched}. 

\begin{table}
\centering
\rowcolors{2}{gray!25}{white}
\begin{tabular}{lp{.7\textwidth}}\toprule
\emph{Policy} & \emph{Description} \\\midrule
\schedfifo & Real-time tasks can run at a minimum of 32 fixed-priorities until they are preempted or yield. \\
\schedrr   & As per \schedfifo but with an added timeslice. If the timeslice for a thread expires, it is added to the tail of the scheduling queue for its priority.\\
\schedsporadic & Specifies sporadic servers as described in \Cref{p:sporadic} and can be used for
    temporal isolation. For practical requirements, the POSIX specification of \schedsporadic
    specifies a maximum number of replenishments which is implementation defined. \\\bottomrule
\end{tabular}
\caption{\gls{POSIX} real-time scheduling policies}
\label{tab:posix-sched}
\end{table}

\citet{Faggioli_08} provides an implementation of \schedsporadic, which \citet{Stanovic_BWH_10}
use to show that the POSIX definition of the sporadic server is incorrect and can allow tasks to
exceed their utilisation bound.  The authors provide a modified algorithm for merging and abandoning
replenishments which fixes these problems, of which corrections to the pseudo code were published in
\citet{Danish_LW_11}.  In further work \citet{Stanovic_BW_11} show that while sporadic servers
provide better response times than polling servers under average load, under high load the overhead
of preemptions due to fine-grained replenishments causes worse response times when compared to
polling servers.  Consequently they evaluate an approach where servers alternate between sporadic
and polling servers depending on load, where the transition involves reducing the maximum number of
replenishments to 1 and merging available refills.

Resource sharing in the \gls{POSIX} \gls{OS} interface is permitted through mutexes, which can be
used to build higher synchronisation protocols.  \Cref{tab:posix-mutex} shows the specified
protocols. 

\begin{table}
\centering
\rowcolors{2}{gray!25}{white}
\begin{tabular}{lp{.7\textwidth}}\toprule
\emph{Policy} & \emph{Description} \\\midrule
\noprioinherit & Standard mutexes that do not protect against priority inversion. \\
\prioinherit  & Mutexes with \gls{PIP} to prevent priority inversion, recall \Cref{sec:pip}. \\
\prioprotect & Mutexes with \gls{HLP} to prevent priority inversion, recall \Cref{sec:hlp}. \\
\bottomrule
\end{tabular}
\caption{\gls{POSIX} real-time mutex policies for resource sharing.}
\label{tab:posix-mutex}
\end{table}

Although \gls{POSIX} provides \schedsporadic which can be used for temporal isolation (however flawed), the intention of the policy is to contain aperiodic tasks. 
Since \schedsporadic allows tasks to run at a lower priority when they have exceeded their allowed budget in any given period, it follows that the locking protocols \prioinherit and \prioprotect would continue to operate -- although the excess time is not billed to the task. 
If a task never releases a locked resource, temporal isolation is completed violated.
As a result, \gls{POSIX} is insufficient for mixed-criticality systems where tasks of different criticalities share resources.
Few \glspl{OS} implement the full \gls{POSIX} standard, however many incorporate features of it, including Linux.

\subsection{Linux}

While Linux cannot be considered a real-time operating system for high criticality applications, it
is frequently used for lower criticality applications with \gls{SRT} demands.  Additionally, Linux
is often used as a platform for conducting real-time systems research. 
% TODO cite completely fair scheduler
Linux has fixed-priority preemptive scheduler which is split into scheduling classes.  Real-time
threads can be scheduled with \gls{POSIX} \schedfifo and \schedsporadic.  best effort threads are
scheduled with \gls{CFS}, and real-time threads are scheduled either \gls{FIFO} or round-robin, and
are prioritised over the best effort tasks.  Fixed priority threads in Linux are completely trusted:
apart from a bound on total execution time for real-time threads which guarantees that best effort
threads are scheduled (referred to as real-time throttling~\citep{Corbet_08}), individual temporal
isolation is not possible.

Linux version 3.14 saw the introduction of an \gls{EDF} scheduling class to Linux~\citep{Corbet_09},
which is between the fair and the fixed priority scheduling classes.  The \gls{EDF} implementation
allows threads to be temporally isolated using \gls{CBS}.

Scheduling in Linux promotes the false correlation we see in many systems: real-time tasks are
automatically trusted (unless scheduled with \gls{EDF}) and assumed to be more important, or more
critical, than best effort tasks.  In reality, criticality and real-time strictness are orthogonal.
Linux does not provide any mechanisms for asymmetric protection beyond priority.

On the resource sharing side Linux provides real-time locking via the POSIX  as per
\Cref{tab:posix-mutex}.
 
\subsection{Real-time Linux Extensions}

A large amount of projects exist that attempt to retrofit more extensive real-time features onto
Linux.  We briefly summarise major and relevant works here.  One of the original
works~\citep{Yodaiken_Barabanov_97} runs Linux as a fully preemptable task via virtualisation and
kernel modifications.  Interrupts are handled by the virtualisation layer, and only directed to
Linux if required.  This means that real-time tasks do not have to suffer from long interrupt
latencies, however it also means that devices drivers need to be rewritten from scratch for
real-time.

\litmus~\citep{Calandrino_LBDA_07} is an extension of Linux that allows for pluggable real-time
schedulers to be easily developed for testing multiprocessor schedulers.

Linux/RK~\citep{Oikawa_Rajkumar_98} is a resource kernel implementation Linux that is often used to
implement new scheduling algorithms.  \gls{ZS} scheduling~\citep{deNiz_LR_09} was implemented and
tested using Linux/RK.

Whilst Linux implementations are suitable for implementing algorithms, being used as test-beds and
even being deployed for non-critical \gls{SRT} applications, ultimately Linux is not a suitable
\gls{RTOS} for running safety-critical \gls{HRT} applications. The large amount of source code
results in a colossal trusted computing base,where it is impossible to guarantee correctness through
formal verification or timeliness through {\gls{WCET}} analysis.  Major reasons for adapting Linux
to real-time are the existing applications and wide array of device and platform support. For
mixed-criticality systems these advantages can be leveraged by virtualising Linux to run \gls{SRT}
and best effort applications.


%TODO{Where does Free RTOS go??}

\section{Microkernels}

Microkernels, as introduced in \cref{sec:background-operating-systems} are small operating systems
kernels which contain the minimum amount of software to implement an OS. Real-time concepts are not
new microkernels. In this section we present several existing mechanisms that exist in microkernels
that can be used to support mixed-criticality systems and survey existing microkernels. 

\subsection{Capabilities}
\label{s:capabilities}

Capabilities~\citep{Dennis_VanHorn_66} are an established mechanism for fine-grained access control
to spatial resources. A capability is a unique, unforgeable token that gives the possessor
permission to access an entity or object in system. Capabilities can have different levels of access
rights, e.g. read, write, execute etc.

\subsection{IPC}
%TODO{describe IPC, how it can be used to make resource servers}
% TODO picture of ipc, to build on and refer to later
% TODO performance of ipc, why its important
% TODO define fastpath here. 


\subsection{Timeslice Donation}

%TODO{diagrams}
%TODO{describe timeslice donation}
%TODO{and migrating threads}



\subsection{Scheduling contexts}

Many kernels distinguish between a threads execution context (the registers, stack etc) and scheduling context (access to processing time) in order to allow the scheduling context to transfer between threads for accounting purposes.
This is a more explicit form of timeslice donation: when the scheduling context is passed between client and server, the scheduling context of the client is billed for activity on the servers behalf. 
Designs differ as to whether scheduling context donation is required or optional. 
Some designs allow multiple execution contexts to be attached to one scheduling context, such that the scheduling context forms a secondary run queue. 
In others, multiple scheduling contexts can be bound to one execution context, allowing the execution context to execute on any scheduling context with available execution time. 

\citet{Steinberg_WH_05} scheduling context consisted of a time quanta and a priority and were donated across \gls{IPC}.

\subsection{Examples}

\paragraph{Real-time Mach}~\citep{Mercer_RZ_94, Mercer_ST_93} first introduced the concept of processor capacity reserves, which is the first form of scheduling contexts in microkernel design.
In Real-time Mach, scheduling context donation was compulsory where processor capacity reserves were used to implement temporal isolation, however threads could be reserved or non-reserved threads. 
Threads which had exhausted their reservation were able to be scheduled by a second level time-sharing scheduler that scheduled all of the unreserved threads, if no unexhausted reservations were present in the scheduler.
Real-time Mach used fixed-priority scheduling and rate-monotonic analysis to conduct an admission test for reservations in the kernel.
If a server exausted the donated reservation, it would finish the operation using the time-sharing scheduler. 
Finally, Real-time Mach provided \gls{PIP} over \gls{IPC} to avoid priority inversion~\citep{Tokuda_NR_90}.

\paragraph{KeyKOS}~\citep{Bomberger_FFHLS_92}

\paragraph{EROS}~\citep{Shapiro_SF_99} is a capability based operating system designed to demonstrate the viability of capability-based systems in terms of performance. 
EROS introduced the idea of a single-use reply capability, termed a \emph{resume} capability, which when invoked would be consumed and allows the receiver to reply to the sender.
EROS also used what is known as an \emph{entry} capability, which allowed the holder to invoke the services provided by a program within a particular process. 
EROS uses the same scheduling and resource sharing policies as Real-time Mach, using capacity reserves, however these are not accessed via the capability system.

\paragraph{DROPS} (Dresden Real-time OPerating System)~\citep{Haertig_BBHHMRSW_98} is an L4 based real-time microkernel that also takes the resource reservation route to temporal isolation.
The scheduling scheme in DROPS allows a process to reserve a priority for an amount of cycles during a time period.
Page colouring is used to reserve parts of the caches for real-time tasks, significantly decreasing upper bounds on \gls{WCET}.

\paragraph{NOVA}~\citep{Steinberg_Kauer_10} is a capability-based microkernel aimed at virtulisation. 
NOVA provides fixed-priority round-robin scheduling, with priority inheritance across IPC~\citep{Steinberg_BK_10}.
Priorities in NOVA reflect importance: it is not a real-time kernel, and periodic scheduling is not available.  
NOVA provides \gls{BWI} through the microkernel mechanism of donation is used, so NOVA does not provide concrete reservations, but when a timeslice expires other tasks waiting for the resource \emph{help} the blocked task, where the blocked task excecutes on the pending tasks timeslice.
If the task holding the resource does not release it, all pending tasks will block forever.


\paragraph{Fiasco.O.C} is an L4 microkernel where scheduling, accounting, enforcement and admission are all implemented in the kernel, with the motivation that these functions would have high overheads if implemented at user-level. 
Resource reservations are realised as \emph{scheduling contexts} which act as timeslices: a budget paired with a priority, and a replenishment rule. 
Some scheduling mechanisms are exposed to the user;~\citet{Lackorzynski_WVH_12} alter the system call interface to support multiple mixed-criticality guests.
They find that the only way to avoid scheduling integrity violations is to export scheduling information to the host \gls{VM}.
Guests can change scheduling contexts on priority switches such that the host schedules guests with enough time to schedule all tasks.
Additionally, guests associate scheduling contexts to \glspl{ISR}.

An \gls{RBED} implementation has also been completed on OKL4~\citep{Petters_LHE_09}.
%TODO{more on OKL4}

\paragraph{seL4} is a microkernel that is particularly suited to safety-critical, real-time systems with one major caveat: the lack of real-time scheduling support.
Three main features of seL4 support this claim: it has been formally verified for correctness~\citep{Klein_EHACDEEKNSTW_09} and other properties~\citep{Sewell_WGMAK_11}; All memory management, including kernel memory, is all at user-level~\citep{Elkaduwe_Derrin_06}; Finally it is the only \gls{OS} to date with full \gls{WCET} analysis~\citep{Blackham_SCRH_11}.
The scheduler in seL4 has been left intentionally underspecified~\citep{Petters_EH_12} for later work.
The current implementation is a placeholder, and follows the traditional L4 scheduling model~\citep{Ruocco_06} --- a fixed-priority, round-robin scheduler with 256 priorities.

\paragraph{\textsc{Minix 3}}~\citep{Herder_BGHT_06} is a traditional microkernel with a focus on reliability rather than performance.
{\sc Minix 3} has been adapted for real-time with temporal isolation provided by resource servers~\citep{Mancina_LFHGT_09}.
 The implementation is designed for bandwidth servers to be implemented at user-level. 
This is achieved by adding system calls allowing for the kernel's best-effort scheduling policy to be switched to \gls{EDF} per process. 
Eight system calls are added, as well as semantics for the kernel to up-call the resource server when a real-time task is blocked, unblocked, exits or exhausts its budget.
Real-time tasks execute as the second-highest priority tasks in the system, the highest being reserved for the bandwidth server. 
The advantage of this implementation is that different types of bandwidth servers can be implemented on top of {\sc Minix 3} without kernel modifications.
 Although not covered in the paper, this could be extended to a split, user-level \gls{RBED} implementation, however the overheads of such an approach are unclear. 
The {\sc Minix 3} implementation is a good example of implementing bandwidth servers with minimal kernel modifications.
 {\sc Minix 3} is not aimed at hard-real time, supporting only mixed-criticality between \gls{SRT} and best effort processes.


\section{Other research kernels}

\paragraph{\composite} is a component-based \gls{OS} with similar with goals to a microkernel, however with a more dominant focus on support for fine-grained components.
In \composite all four mechanisms are implemented at user-level, however unlike a microkernel, \composite contains device drivers inside the kernel.

A hierarchical scheduling framework, \hires\citep{Parmer_West_11}, is used in \composite.
\hires delivers timer interrupts to a root, user-level scheduling component, which are then forwarded through the hierarchy to child schedulers.
Consequently, scheduling overhead increases as the hierarchy deepens.
Child schedulers with adequate permissions use a dedicated system call to tell the kernel to switch threads.
The kernel itself does not provide blocking semantics, which are also provided by user-level schedulers.
This design offers total scheduling policy freedom, as user-level scheduling components can implement all of the goals of a resource kernel according to their own policy.
 
% TipToe what is it and how does it schedule
\paragraph{Tiptoe}~\citep{Craciunas_KPRS_09} is a now-defunct research microkernel that also aims at temporal isolation between user-level processes and the operating system.
Like {\sc Minix 3}, Tiptoe uses the \gls{CBS} approach, although it uses variable bandwidth servers which allow for bandwidths to be altered.
Tiptoe implements bandwidth servers inside the kernel.
%TODO{look into tiptoe more detail on how it handles resource sharing}

\paragraph{Barrelfish}~\citep{Peter_SBBIHR_10} is a capability-based multi-kernel \gls{OS}, where a separate kernel runs on each processing core and kernels themselves share no memory and are essentially \emph{\gls{CPU}-drivers}.
Each CPU-driver is single-threaded and non-preemtible. 
 , although not explicitly real-time, implements a version of \gls{RBED} for managing distributed \gls{SRT} processes.
Barrelfish schedules dispatchers, which are roughly equivalent to processes. 
Execution context and scheduling context are not split.
When messages are sent between dispatchers on barrelfish the sender can choose to yield to the receiver or not by flags on the message.

TODO Quest first, then quest v. 
As a separation kernel, Quest does not provide resource sharing mechanisms by definition. 

\paragraph{Quest-V}~\citep{Danish_LW_11} is a separation kernel / hypervisor with fixed priority scheduling.
Temporal isolation is provided by sporadic servers, however I/O and normal processes are
ditinguished statically: I/O processes use polling servers and normal processes use sporadic
servers, with a maximum replenishment length of 32. In Quest-V, separate partitions are assigned to
different priority levels, and are not permitted to communicate. 

\subsection{Commercial RTOSes}
%TODO add details on resource sharing to this section
There are several widely deployed \glspl{RTOS} used commercially.  The majority implement part or
all of \gls{POSIX}.  We present three popular alternatives: VxWorks, QNX Neutrino and RTEMS.

\paragraph{RTEMS}~\citep{RTEMS:URL} is an open-source \gls{RTOS} that operates with or without
memory protection, although in either case it is statically configured.  Although it is an open
source project, RTEMS is used widely in industry and research.  The main scheduling policy is
\gls{FPRM}, however \gls{EDF} is also available with temporal isolation an option using \gls{CBS}.
No temporal isolation mechanisms are present for fixed-priority scheduling.  RTEMS provides
semaphores with priority inheritance or immediate priority ceilings for resource sharing, as well as
higher level primitives for these. 

\paragraph{QNX Neutrino}~\citep{QNX_10} is a commercial, microkernel-based separation kernel that provides
\gls{FP} scheduling and resource sharing with POSIX semantics.  QNX satisfies many industry
certification standards, although these in practice do not require {\gls{WCET}} analysis or formal
verification of correctness.

\paragraph{VxWorks}~\citep{VxWorks_08} is a monolithic \gls{RTOS} deployed most notably in aircraft
and spacecraft.  It supports \gls{FP} scheduling with a native POSIX-compliant scheduler.  VxWorks
also has a pluggable scheduler framework, allowing developers to implement their own, in-kernel
scheduler.

\paragraph{PikeOS}~\citep{PikeOS:URL} implements ARINC 653~\citep{ARINC653} via a separation
microkernel which paravirtualises RTOS guests in different partitions.

There are many other \gls{RTOS}es used commercially, including \citet{Deos:URL} but the general pattern is POSIX-compliant, \gls{FP} scheduling and resource sharing.
This brief survey shows that \gls{FP} scheduling is dominant in industry due to its place in the POSIX standard. 
Although temporal isolation is sometimes provided with the possibility of bounded bandwidth via
\glspl{CBS}, asymmetric protection is not provided.  In addition to scheduling, code-bases are
generally large and complex, and beyond the grasp of modern {\gls{WCET}} analysis.  Although all of
these \gls{RTOS}es are deployed in safety critical systems, their support for mixed-criticality
applications is questionable if non-existent. 



\subsection{Summary}

%TODO{what did we learn in this chapter?}
%TODO{note gap between rt theory and systems}
%TODO{note the false correlation of priority, rt model, trust, criticality etc}

\begin{table}
\centering
\rowcolors{3}{}{gray!25}
\begin{tabular}{lllll}\toprule
  \emph{OS} & \emph{Scheduler}  & \emph{Isolation} & \emph{Asymmetric} & \emph{Resource}\\
            &                   &                  & \emph{protection} & \emph{Sharing}\\\midrule
Linux       & \gls{FP} + \gls{EDF} & \gls{CBS}          & \no  & \gls{PIP}, \gls{HLP}\\
POSIX       & \gls{FP}             & \gls{SS}           & \no  & \gls{PIP}, \gls{HLP} \\
Linux/RK    & TODO                 & TODO               & \no  & TODO \\
RTEMS       & \gls{FP} + \gls{EDF} & \gls{CBS}          & \no  & \gls{PIP}, \gls{HLP}\\
QNX         & \gls{FP}             & \gls{SS}           & \no  & TODO \\
VxWorks     & \gls{FP}             & TODO               & \no  & TODO  \\
Real-Time Mach & \gls{FP}          & \gls{DS}           & \no  & \gls{PIP} \\
EROS        & \gls{FP}             & \gls{DS}           & \no  & \gls{PIP} \\
Minix 3     & TODO                 & TODO               & \no  & TODO \\
Tiptoe      & TODO                 & TODO               & \no  & TODO   \\
Barrelfish  & \gls{EDF}            & \gls{RBED}         & \no  & Timeslice donation   \\
DROPS       & \gls{FP}             & TODO               & \no  & TODO \\
NOVA        & \gls{FP}             & \no                 & \no  & \gls{BWI} \\
Quest-V     & \gls{FP}             & \gls{SS}           & \no  & \no \\
seL4        & \gls{FP}             & \no                 & \no  & \no \\
Composite   & user level           & TODO               & TODO & TODO \\
\bottomrule
\end{tabular}
\label{t:os-summary}
\caption{Summary of \gls{MCS} support available in existing operating systems.}
\end{table}

