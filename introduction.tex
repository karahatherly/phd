\chapter{Introduction}

% motivate

Criticality of a software system refers to the severity of the impact of a failure. 
In a high-criticality system, failure risks significant loss of life or damage to the environment.
In a low-criticality system, failure may risk a downgrade in user-experience.
Traditionally, systems of different criticality were isolated by hardware.
This approach is no longer practical as it has proven inefficient and restrictive.
The result is \emph{mixed-criticality} systems, where software applications with different criticalities execute on the same hardware.

Mixed-criticality systems are more effective and can achieve more than traditional physically isolated systems.
They are more efficient purely from a weight perspective: consolidating functionality onto less physical hardware (and in turn less cabling) corresponds to a direct saving in weight, which has large impacts on the fuel consumption of automobiles and trains, and increases the range of aircraft.
The aviation industry recognised this in the 90s, and consequently developed the \citet{ARINC653} standard.
\citet{ARINC653} allows for sharing of hardware between software applications of mixed criticality, and outlines the primitives required from \glspl{RTOS} built for these systems, mandating full temporal and spatial isolation of the applications.
A decade later the \gls{MCAR}\citep{Barhorst_BBHPSSSSU_09} program was launched, which recognises that in order to construct fully autonomous systems, critical and less critical systems must be able to share resources, which is prevented by \citet{ARINC653}.

%% FIXME you can motivate this with MCAR, camkes
Much research activity has taken place as a result of the \gls{MCAR} program, especially with regard to scheduling theory.
However, the implications for \glspl{RTOS} and the software that runs on them are still open for exploration.
As a result, the goal of this project is to develop a mixed-criticality \gls{RTOS} that allows for full-system schedulability analysis, including any time taken by the \gls{RTOS} itself.
This will involve developing bounded real-time mixed-criticality scheduling and resource sharing mechanisms that are amenable to verification and minimise policy required by the OS.
The implementation platform will be the \selfour microkernel, which is has been designed for the high-assurance, safety-critical domain.

Concisely, the goals of this research are to provide:
\begin{enumerate}[label=\textbf{G\arabic*}]
\item A principled approach to processor management, treating time as a fundamental kernel resource, while allowing it to be overbooked, a key requirement of mixed-criticality systems;
\item safe resource sharing between applications of different criticalities and different temporal requirements. 
\end{enumerate}

\paragraph{Outline}

This report is structured as follows:
% TODO update
First, the rest of the current chapter will further motivate and establish the problem.
In \Cref{chap:background} we will explore real-time scheduling theory and it's application to mixed-criticality operating systems, along with \selfour design and a survey of current real-time and mixed-criticality \glspl{OS}.
\Cref{chap:scheduling} presents the our approach and reasoning for our scheduling API mechanisms, 
followed by \Cref{chap:resource-sharing} will explore the background and our model for sharing resources between applications of different criticalities. 
\Cref{chap:implementation} will outline the implementation in detail, and finally
\Cref{chap:evaluation} will present microbenchmarks, full system benchmarks and case studies and results with which we evaluate our design.

\section{Motivation}
\label{sec:motivation}

The criticality of a system reflects the severity of failure, where higher criticality implies higher severity.
Table \ref{tab:criticality_table} shows criticality levels considered when designing software for commercial aircraft in the United States.

\begin{table}
	\begin{center}
		\begin{tabular}{ | l | p{10cm} |} \hline
			\textbf{Level} & \textbf{Impact} \\ \hline
			Catastrophic   & Failure may cause a crash \\ \hline
			Hazardous      & Failure has a large negative impact on safety or performance \\ \hline
			Major          & Failure is significant, but less than hazardous (passenger discomfort or increased crew workload) \\ \hline 
			Minor          & Failure is noticeable (passenger inconvenience or changed route) \\ \hline
			No Effect      & No impact on safety, aircraft operation or crew workload \\ \hline
		\end{tabular}
		\caption{Criticality levels from DO-178B, a safety standard for commercial aircraft.}
		\label{tab:criticality_table}
	\end{center}   
\end{table}    

Systems of different criticality levels require higher engineering and certification standards.
Traditionally, systems of different criticalities were completely isolated: separate physical hardware, with a defined air gap between them. % TODO citation?

While physical separation guarantees isolation, there are downsides, which can be summarise as \gls{SWaP}, functionality, and efficiency.

\paragraph{Size, Weight, Power and Cost}

First, increased physical systems require increased physical resources: increasing production costs and environmental impact, further more increasing the \gls{SWaP} of those systems.
For transporation vehicles, especially aircraft, this goes further to reduce their function: the heavier the system, the more fuel it requires to travel -- increasing \gls{SWaP} increases cost and reduces utility: the system simply cannot do as much with the work required. 
For vehicles, it means their core function is impaired: their transportation range is reduced.

Given the increased amount of computing in every part of our daily lives, the practice of physical isolation has resulted in an unscalable growth in the amount of computing hardware in embedded systems, with some modern cars containing over 100 processors~\citep{Hergenhan_Heiser_08}.

\paragraph{Function}

Additionaly, isolated physical systems cannot interact by definition, which limits the function of those systems. 
\glspl{UAV} and other autonomous vehicles provide an excellent motivations for systems where sub systems of different criticality share resources: meaning they must share hardware.
For example, consider the system for a self driving car as with components as outlined in \Cref{tab:self-driving-car}.
The safety systems are the most critical: if airbags or anti-lock breaks fail this could cause great injury or death to the passengers.
The communications system is least critical, however userful: it downloads weather and road conditions, status of road works and accidents. 
This feeds in to the more critical navigation system, requiring resource sharing.
This sort of system would not be possible without a mixed criticality system: unless the communications system were ceritified to the same level as the navigation system, which would greatly increase the cost of development. 
Consequently, mixed-criticality systems provide increased functionality over physically separated, isolated systems.

% TODO would be cool if this was a real system
\begin{table} 
\begin{center}
\begin{tabular}{|c|c|c|}\hline
    \textbf{System}     & \textbf{Purpose} & \textbf{Criticality} \\\hline
     Airbags            & Safety &  Catastrophic \\\hline
     Anti lock breaks   & Safety &  Catastrophic \\\hline
     Obstacle detection & Safety &  Hazardous \\\hline
     Navigation         & Route planning & Minor \\\hline  
     Communications     & Optimal route planning & No effect \\\hline
\end{tabular}
\caption{Example (made up) systems in a self driving car.}
\label{tab:self-driving-car}
\end{center}
\end{table}

% TODO present UAV case study here, brief list of components with criticalities - expand and build upon throughout document

% An example of a mixed-criticality system is a modern car.
% \gls{HRT} systems include airbag controls, tyre pressure monitors and anti-lock braking, whereas \gls{SRT} systems include the infotainment unit, electric window controls and air-conditioning.

\paragraph{Efficiency}

High criticality systems are usually \emph{overprovisioned}: both with redundant hardware, and hardware with capacity to deal with the worst possible case: this means that much of the time, excess hardware capacity is unused. 
By leveraging some of this excess capacity for lower criticality tasks, more efficient use of that hardware can be made.

\paragraph{Mixed-Criticality Systems}

A \emph{mixed criticality} system involves multiple applications of different criticalities executing on the same physical hardware. 
Given their improvements to \gls{SWaP}, function and efficiency, mixed-criticality systems offer great advantages over the traditional physical isolation approach. 
% TODO diagram that summarises efficiency 
Such systems have strong requirements:

\begin{itemize}
\item Temporal and spatial isolation are essential to providing the same guarantees as their physically isolated counterparts.
\item Asymetric protection across criticalities is required, which means that higher criticality sub-systems are prioritised over lower ones: if success of a high criticality task may cause a failure in a low criticality task this is permitted, but not vice versa.
\item Trusted resource sharing protocols, which allow tasks of different criticalities to safely share resources.
\end{itemize}

A mixed-criticality operating system then must provide these mechanisms. 


\section{Scope}

This thesis focuses on mechanisms for building mixed-criticality systems. 
We focus on uniprocessor and \gls{SMP} systems with \glspl{MMU} for ARM and x86.
We do not consider side- or timing-channels as part of this research, and as that is entire research field in itself.

\section{Contributions}

% TODO some contributions





