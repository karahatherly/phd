\chapter{Introduction}

% what is criticality
\emph{Criticality} of a software system refers to the severity of the impact of a failure.  In a
high-criticality system, failure risks significant loss of life or damage to the environment.  In a
low-criticality system, failure may risk a downgrade in user-experience.  Traditionally, systems of
different criticality were isolated by hardware.  This approach is no longer practical as it has
proven inefficient and restrictive.  The result is \emph{mixed-criticality} systems, where software
applications with different criticalities execute on the same hardware.

% introduce challenges, example
However, mixed-criticality systems have conflicting requirements that challenge \glspl{OS} design:
they require mutually distrusting components of different criticalities to share resources and must
degrade gracefully in the face of failure.  For example, an \gls{AAV} has multiple inputs to its
flight-control algorithm: object detection, to avoid flying into obstacles, and navigation, to get
to the desired destination.  Clearly the object detection is more critical than navigation, as
failure of the former can be catastrophic, while the latter would only result in a non-ideal route.
Yet the two subsystems must cooperate, accessing and modifying shared data, thus cannot be fully
isolated.

The \gls{AAV} is an example of a mixed-criticality system, a notion that originates in avionics and its
need to reduce \gls{SWaP}, by consolidating growing functionality onto a smaller
number of physical processors. More recently the
\gls{MCAR}\citep{Barhorst_BBHPSSSSU_09} program was launched, which recognises that in order to
construct fully autonomous systems, critical and less critical systems must be able to share
resources. However, resource sharing between components of mixed-criticality is heavily restricted by current
standards.

% what's wrong with static partitioning
While MCS are becoming the norm in avionics, this is presently in a very restricted form: the system
is orthogonally partitioned spatially and temporally, and partitions are scheduled with fixed time
slices~\citep{ARINC653}. Any components that share resources are promoted to the highest criticality
level and must meet the same certification standards; in other words, a critical component must not
trust a lower criticality one to behave correctly. This limits integration and cross-partition
communication, and implies long interrupt latencies and poor resource utilisation. 
% not just planes
These challenges are not unique to avionics: top-end cars exceeded 70 processors ten years
ago~\citep{Broy_KPS_07}; with the robust packaging and wiring required for vehicle electronics, the
SWaP problem is obvious, and will be magnified by the move to more autonomous operation. Other
classes of cyber-physical systems, such as smart factories, will experience similar challenges.

% more than consolidation
However, mixed-criticality systems are about more than basic consolidation; they can achieve more
than traditional physically isolated systems. Allowing untrusted, less critical components to
share hardware and communicate with more critical components, far more complex software can be
introduced to the system. Examples include heuristic algorithms that are common in artificial
intelligence algorithms, or internet connected software which by its nature cannot be completely
trusted. Both of these use-cases are far too complex to certify to the highest criticality standard,
but are essential for emerging cyber-physical systems like self-driving cars.

% finally, overcommitting
Our goal is to design and implement an OS that provides the right mechanisms for efficiently
supporting mixed criticality systems, and to reason about their safety.
The implementation platform will be the \selfour~\citep{Klein_EHACDEEKNSTW_09}.
microkernel, which is has been designed for the high-assurance, safety-critical domain.

Concisely, the goals of this research are to provide:
\begin{enumerate}[label=\textbf{G\arabic*}] 
    \item\label{G1} A principled approach to
    processor management, treating time as a fundamental kernel resource, while
    allowing it to be overbooked, a key requirement of mixed-criticality systems;
    \item safe resource sharing between applications of different criticalities and
    different temporal requirements.  
\end{enumerate}


\section{Motivation}

% expand on notion of criticality with more detail and examples. Give enough background to justify contributions.
As noted in the introduction, the \emph{criticality} of a system reflects the
severity of failure, where higher criticality implies higher severity.  Table
\ref{tab:criticality_table} shows criticality levels considered when designing
software for commercial aircraft in the United States.

\begin{table}
     \centering
     \rowcolors{2}{gray!25}{}
     \begin{tabular}{ l p{10cm}} \toprule
         \emph{Level}   & \emph{Impact} \\ \midrule
         Catastrophic   & Failure may cause multiple fatalities, usually with loss of the airplane. \\
         Hazardous      & Failure has a large negative impact on safety or performance, or reduces the
                          ability of the crew to operate the aircraft due to physical distress or 
                          a higher workload, or causes serious or fatal injuries among the passengers.\\
         Major          & Failure significantly reduces the safety margin or significantly increases
                          crew workload. May result in passenger discomfort (or even minor
                          injuries).\\
         Minor          & Failure slightly reduces the safety margin or slightly increases crew
                          workload. Examples might include causing passenger inconvenience or a
                          routine flight plan change. \\
         No Effect      & Failure has no impact on safety, aircraft operation or crew workload. \\
         \bottomrule
     \end{tabular}
     \caption{Criticality levels from DO-178C, a safety standard for commercial aircraft.}
     \label{tab:criticality_table}
 \end{table}

% TODO are there any citations on cost per line of code?
Higher engineering and safety standards are required for higher criticality levels, up to
certification by independent certificate authorities at the highest levels. As a result, highly
critical software is incredibly expensive to develop and tends to have low complexity in order to
minimise costs. Any software that is not fully isolated from a critical component is promoted to
that level of criticality, increasing the production cost.

Traditionally, systems of different criticality levels were fully isolated with
air gaps between physical components. However, given the increased amount of
computing in every part of our daily lives, the practice of physical isolation
has resulted in unscalable growth in the amount of computing hardware in
embedded systems, with some modern cars containing over 100
processors~\citep{Hergenhan_Heiser_08}. The practise of physical separation is no 
longer viable for three reasons: \gls{SWaP}, efficiency, and function.

\paragraph{SWaP} First, systems with air-gaps require increased physical
resources: increasing production costs and environmental impact.
For vehicles, especially aircraft, this goes further to reduce their function; the
heavier the system,  the more fuel it requires to travel. This in turn reduces
utility in the form of reducing vehicle range. Therefore the consolidation that comes
with combining systems of different criticalities is worthwhile.  

\paragraph{Function} 
Mixed-criticality systems also bring opportunities for new types of
systems, and are indeed required for emerging cyber-physical systems like
advanced driver assistance systems, autonomous vehicles and internet of things devices.
For example consider the system for a self driving
car as with components as outlined in \Cref{tab:self-driving-car}.  The safety
systems are the most critical: if air bag or anti-lock brakes fail this could
cause great injury or death to the passengers.  The communications system is
least critical, however useful: it downloads weather and road conditions,
status of road works and accidents.  This feeds in to the more critical
navigation system, requiring resource sharing.  This sort of system would not
be possible without a mixed criticality system: unless the communications
system were certified to the same level as the navigation system, which would
greatly increase the cost of development.  Consequently, mixed-criticality
systems provide increased functionality over physically separated, isolated
systems.

% TODO would be cool if this was a real system
\begin{table} 
\rowcolors{2}{gray!25}{}
\centering
\begin{tabular}{ccc}\toprule
    \emph{System}     & \emph{Purpose} & \emph{Criticality} \\\midrule
     Airbags            & Safety &  Catastrophic \\
     Anti lock breaks   & Safety &  Catastrophic \\
     Obstacle detection & Safety &  Hazardous    \\
     Navigation         & Route planning & Minor \\  
     Communications     & Optimal route planning & No effect \\
    \bottomrule
\end{tabular}
\caption{Fictional example systems in a self driving car.}
\label{tab:self-driving-car}
\end{table}

% TODO present UAV case study here, brief list of components with criticalities - expand and build upon throughout document

% An example of a mixed-criticality system is a modern car.
% \gls{HRT} systems include airbag controls, tyre pressure monitors and anti-lock braking, whereas \gls{SRT} systems include the infotainment unit, electric window controls and air-conditioning.

\paragraph{Efficiency}
\label{p:arinc}

High system utilisation is essential for addressing SWaP challenges, and high responsiveness is
important for much of a system's desirable functionality. But high utilisation is a challenge in
critical real-time systems, which are usually over-provisioned, both with redundant hardware and 
excess capacity; the core \emph{integrity property} is that deadlines must always be
met, meaning that there must always be time to let such threads execute their full \emph{\gls{WCET}}.
This may be orders of magnitude larger than the typical execution time, and
computation of safe \gls{WCET} bounds for non-trivial software tends to be highly pessimistic
\citep{Wilhelm_EEHTWBFHMMPPSS_08}.  

Consequently, most of the time the highly-critical components
leave plenty of slack, which should be usable by less critical components. In terms of
schedulability analysis, this constitutes an \emph{overcommitted} system, where not everything is
guaranteed to be schedulable.  In case of actual overload, the system must guarantee sufficient time
to the critical components at the expense of the less critical ones, which is referred to as
\emph{asymmetric protection}.

Additionally, the higher the criticality
of a software component, the higher the pessimism in analysis, raising the \gls{WCET} and reducing
the amount of software that can be consolidated onto a single platform. Therefore it is 
impractical to raise all software to the same criticality level should it share hardware, as the
total resource utilisation is lowered.

More complex systems such as those found in aviation allow for highly
restricted mixed-criticality systems in the form of separation kernels.
\citet{ARINC653} allows for sharing of hardware between software applications
of mixed criticality, and outlines the primitives required from an \gls{OS}
built for these systems, mandating full temporal and spatial isolation of the
different criticality components and controlled communication. However, this is one 
standard that may not be suitable for all types of system.

\section{Contributions}

Given their improvements to \gls{SWaP}, function and efficiency, mixed-criticality systems offer
great advantages over the traditional physical isolation approach. 
\citet{Ernst_DiNatale_16} identify two sets of mechanisms that need to be provided in order to
support true mixed-criticality systems:

\begin{enumerate}
    \item kernels and schedulers that guarantee resource
        management to provide independence in the
        functional and time domain; separation kernels
        are the most notable example;
    \item mechanisms to detect timing faults and control
        them, including monitors, and the scheduling
        provisions for guaranteeing controllability in the
        presence of faults.
\end{enumerate}

The contributions of our research are to provide the core mechanisms for building whole
mixed-criticality systems upon. 

\section{Approach}

In this thesis we look to systems and real-time theory related to scheduling, resource allocation
and sharing, criticality and trust to develop a set of mechanisms required in an \gls{OS} to 
support mixed criticality systems. We implement those mechanisms in \selfour and develop a set of
microbenchmarks and case studies to evaluate the implementation. 

\Cref{chap:background} establishes the basic terminology required to present
the remaining ideas. In \Cref{chap:scheduling} we examine in detail methods for achieving temporal
isolation and safe resource sharing in real-time systems; \Cref{chap:operating-systems} investigates
the same concepts from the systems perspective, and presents a survey of existing commercial, open-source
and research operating systems. We then present the relevant details of our implementation platform,
\selfour, in \Cref{chap:sel4}.

We then draw on all of the previous chapters to design mechanisms for efficiently supporting mixed
criticality systems, and present our findings, along with the trade-offs and consequences of our
design in \cref{chap:model}.  \Cref{chap:implementation} delves deeply into the implementation details.
Finally \Cref{chap:evaluation} presents our benchmarks, case studies and findings. 

\section{Scope}

This thesis focuses on mechanisms for building mixed-criticality systems. 
We focus on uniprocessor and \gls{SMP} systems with \glspl{MMU} for ARM and x86.
We do not consider side- or timing-channels as part of this research.
% TODO is there anything else that needs to be mentioned here
